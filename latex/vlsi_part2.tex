\documentclass[11pt,center]{beamer}

\usepackage{fancybox}
\usepackage{graphics}
\usepackage{spot}
\usepackage{tikz}
\usepackage{minibox}
\usetikzlibrary{arrows}
\usepackage[absolute,overlay]{textpos}
\usetheme{metropolis}

\usepackage[export]{adjustbox}

\definecolor{light-gray}{gray}{0.86}
\title{\huge{Adaptive Machine Learning}}
\author{Βασίλειος Αταλόγλου \\ Κωνσταντίνος Σαμαράς-Τσακίρης}
\date{\today}

\begin{document}

  \begin{frame}%{\maketitle}
	  \titlepage
  \end{frame}



\section{Προσομοιώσεις - Αποτελέσματα}

	\begin{frame}{Τεχνητά δεδομένα}
		Δημιουργία set ακολουθιών:
		\begin{itemize}
			\item[--]Απλή πρόβλεψη (A)
			\item[--]Πολλαπλή πρόβλεψη (B)
		\end{itemize}
		\vfill
		\includegraphics[width=0.6 \textwidth,center]{../pics/sequences.jpg}
		\vfill
		\pause
		Προσθήκη θορύβου ανάμεσα στις ακολουθίες
	\end{frame}



	\begin{frame}{Τεχνητά δεδομένα}
		Σύγκριση διαφορετικών υλοποιήσεων για μονή πρόβλεψη:\\
		\vspace{1em}
		\includegraphics[width=0.7 \textwidth,center]{../pics/single_prediction.jpg}
		\pause
		\begin{tabbing}
  			Το HTM:  \=α) πετυχαίνει perfect accuracy\\
  			\>β) προσαρμόζεται στις αλλαγές του περιβάλλοντος\\
  		\end{tabbing}
	\end{frame}

	\begin{frame}{Τεχνητά δεδομένα}
		Πολλαπλές προβλέψεις:
		\vfill
		\includegraphics[width=0.7 \textwidth,center]{../pics/multiple_predictions.jpg}
		\pause
		\vfill
		To HTM είναι το \alert{μόνο} που πετυχαίνει perfect accuracy.\\
		\pause
		Αιτία: Αναπαράσταση με \textbf{SDR} !!
	\end{frame}

	\begin{frame}{Τεχνητά δεδομένα}
		\begin{columns}
		\column {0.4 \textwidth}
		\\
			\visible<1->{
			Πρόβλεψη high-order ακολουθιών} \\
			\vspace{5em}
			\visible<2->{
			Ανθεκτικότητα σε κατεστραμμένο δίκτυο}
		\column {0.6 \textwidth}
			\vspace{1em}
			\visible<1->{
			\includegraphics[width=0.6 \textwidth,left]{../pics/sequence_order.jpg}}
			\vspace{1em}
			\visible<2->{
			\includegraphics[width=0.6 \textwidth,left]{../pics/cell_death.jpg}}
		\end{columns}
	\end{frame}

	\begin{frame}{Πραγματικά δεδομένα}
		Ζήτηση ταξί Νέας Υόρκης (διάστημα 30 λεπτών)
		\includegraphics[width=0.5 \textwidth,center]{../pics/taxi_demand.jpg}
		\vfill
		\pause
		\begin{columns}
		\column {0.5 \textwidth}
			Στόχος: Πρόβλεψη της ζήτησης 2.5 ώρες πριν
		\column {0.4 \textwidth}
			\vspace{1em}
			\includegraphics[width=0.6 \textwidth,left]{../pics/taxi_accuracy.jpg}
		\end{columns}
	\end{frame}

	\begin{frame}{Πραγματικά δεδομένα}
		Προσαρμογή σε αλλαγή των δεδομένων εισόδου\\
		\includegraphics[width=0.6 \textwidth,center]{../pics/taxi_adaptation.jpg}
		\vfill
		\pause
		Το HTM κατάφερε να μάθει το νέο μοντέλο σε 2 εβδομάδες!

	\end{frame}


\section{Hardware}


	\begin{frame}{SpiNNaker Chip}
		Yψηλής παραλληλοποίησης υπολογιστικό σύστημα για την μοντελοποίηση και προσομοίωση spiking 				neural networks.
		\pause
		\vfill
		Βασικό στοιχείο το SpiNNaker Chip Multiprocessor (CMP):
		\vfill
		\begin{columns}
			\column{0.5 \textwidth}
			\includegraphics[width=0.75 \textwidth,right]{../pics/CMPspecs.jpg}
			\column{0.5 \textwidth}
			\includegraphics[width=0.75 \textwidth,left]{../pics/CMP.jpg}
		\end{columns}
	\end{frame}

	\begin{frame}{SpiNNaker Network}
		SpiNNaker: Πίνακας από κόμβους που περιέχουν CMP και 128MB SDRAM σε κοινό package.
		\pause
		\vfill
		Συνολικά, έχουμε:
		\begin{columns}
			\column{0.6 \textwidth}
				\begin{itemize}
					\item[--] 57600 CMPs
					\item[--] $10^6$ ARM968
					\item[--] $10^{9}$ νευρώνες ($1 \%$ εγκεφάλου)
					\item[--] 228 TIPS
					\item[--] 90KW ισχύς
				\end{itemize}
			\column{0.5 \textwidth}
				\includegraphics[width=0.9 \textwidth,left]{../pics/Spinnaker.jpg}
		\end{columns}
	\end{frame}

	\begin{frame}{Χαρακτηριστικά SpiNNaker}
		\begin{description}[Sensor output]
			  \item[Επικοινωνία] Address Event Representation (AER), NoC
			  \pause
			  \vfill
			  \item[Μνήμη] Fast-access για την κατάσταση του νευρώνα\\
			  SDRAM για την κατάσταση των συνάψεων
			  \pause
			  \vfill
			  \item[Κατανάλωση] Ασύγχρονη Επικοινωνία\\
			  Sleep mode στην idle κατάσταση \\
		\end{description}
	\end{frame}

	\begin{frame}{Παράδειγμα υλοποίησης}
		Σύστημα για classification χειρόγραφων ψηφίων (MNIST database) μέσω Deep Neural Network
		\vfill
		\pause
		\begin{columns}
		\column {0.4 \textwidth}
			\begin{itemize}
			\item<2->[--] Ακρίβεια
			\item<3->[--] Ισχύς: 0.3 Watt
			\item<4->[--] Latency
			\end{itemize}

		\column {0.6 \textwidth}
		\visible<2->{
			\includegraphics[width=0.6 \textwidth,left]{../pics/Accuracy.jpg}
		}
			\vspace{1em}
		\visible<4->{
			\includegraphics[width=0.6 \textwidth,left]{../pics/Latencies.jpg}
		}
		\end{columns}



	\end{frame}

	\begin{frame}{BrainScaleS Project (BSS)}
	Υβριδική πλατφόρμα: Cluster + Νευρομορφικό σύστημα
	\vfill
	\pause
	Βασικό στοιχείο: High Input Count Analog Neural Network (HICANN)
	\vfill
	\includegraphics[width=0.4 \textwidth,center]{../pics/HICANN.jpg}
	\pause
	\vfill
	Mixed-signal: Αναλογικά νευρωνικά κυκλώματα, Ψηφιακή επικοινωνία
	\end{frame}

	\begin{frame} {HICANN chip}
		Κάθε HICANN chip:
		\begin{itemize}
			\item[--] 512 νευρώνες τύπου AdEx
			\item[--] 2 ομάδες 226 συνάψεων (πχ. proximal, distal)
		\end{itemize}
		\pause
		\vfill
		Ένα wafer περιέχει $364$ HICANN cores, δηλαδή:
		\begin{itemize}
			\item[--] $200\cdot 10^3$ νευρώνες
			\item[--]$45\cdot 10^6$ συνάψεις
		\end{itemize}
		\pause
		\vfill
		Η σημερινή πλατφόρμα αποτελείται 20 wafers.

	\end{frame}

	\begin{frame} {HICANN Chip}
		\begin{columns}
		\column {0.5 \textwidth}
			\begin{itemize}
				\item<1->[--]Motherboard πάνω\\ από το wafer
				\item<1->[--]FPGA για inter-wafer επικοινωνία
			\end{itemize}
			\vspace{5em}
			\begin{itemize}
				\item<2->[--]Διαχωρισμός σε reticles
				\item<2->[--]Συνδέσεις μέσω πυκνού δικτυώματος από wires
			\end{itemize}

		\column {0.5 \textwidth}
			\visible<1->{
			\includegraphics[width=0.7 \textwidth,left]{../pics/wafer.jpg}}
			\vspace{1em}
			\visible<2->{
			\includegraphics[width=0.8 \textwidth,left]{../pics/reticle.jpg}}
		\end{columns}
	\end{frame}

	\begin{frame} {HICANN Chip}
		\begin{columns}
		\column {0.5 \textwidth}
			\begin{itemize}
				\item<1->[--] Κυκλώματα μεμβράνης
				\item<2->[--] Είσοδος από synapse driver
				\item<3->[--] Strobe lines
				\item<4->[--] Neuron builder
			\end{itemize}
		\column {0.6 \textwidth}
			\visible<1->{
			\includegraphics[width=0.9 \textwidth,left]{../pics/insideHICANN.jpg}}
			\end{columns}
	\end{frame}

	\begin{frame}{Συνάψεις}
			\begin{itemize}
				\item<1->[--]4 bit SRAM για το βάρος
				\item<1->[--]Ρεύμα ανάλογο του βάρους
				\item<1->[--]MUX για επιλογή εισόδου (excitatory/inhibitory)
			\end{itemize}
		\includegraphics[width=0.8 \textwidth,center]{../pics/synapses.jpg}
	\end{frame}

	\begin{frame}{Υλοποίηση Spatial Pooler}
		Time - based σύστημα: Οι νευρώνες με το μεγαλύτερο input ενεργοποιούνται πρώτοι\\
		\vspace{2em}
		\includegraphics[width=0.5 \textwidth,center]{../pics/spatial_hardware.jpg}
		\pause
		\vfill
		\begin{tabbing}
  			Το inhibition cell(I):  \= α) διατηρεί την πυκνότητα\\
  			\> β) ελέγχει τη σταθερότητα\\
  		\end{tabbing}
	\end{frame}

	\begin{frame}{Υλοποίηση Temporal Pooler}
		\begin{columns}
		\column {0.6\textwidth}
			\begin{itemize}
				\item<1->[--] Ένα κελί (P) για το proximal input
				\item<1->[--] Μια τριάδα (D,I,S) για κάθε νευρώνα του multicolumn
				\vspace{1em}
				\item<2->[--] To Distal (D) αθροίζει τις συνάψεις "πρόβλεψης"
				\vspace{1em}
				\item<3->[--] Το Inhibition (I) καθορίζει ποιοι νευρώνες του ενεργού multicolumn 							ενεργοποιούνται
			\end{itemize}

		\column {0.4\textwidth}
			\includegraphics[width=1 \textwidth,center]{../pics/temporal_hardware.jpg}
		\end{columns}

	\end{frame}

	\begin{frame}{Αποτελέσματα προσομοιώσεων}


	\end{frame}


\end{document}
